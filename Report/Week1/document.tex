\documentclass[a4paper,12pt,oneside]{article}
\usepackage[utf8]{vietnam}
\usepackage{graphicx}
\graphicspath{ {figures/} }
\usepackage{fancyhdr}
\usepackage{geometry}
\usepackage{array}
\usepackage{float}
\usepackage[bottom]{footmisc}
\usepackage{afterpage}
\usepackage{makecell}
\usepackage{multirow}
\usepackage{pdflscape}
\usepackage[table]{xcolor}
\usepackage{indentfirst}%thut dau dong doan moi
\usepackage[unicode]{hyperref}%tao bookmark
\hypersetup{urlcolor=blue,linkcolor=black,citecolor=black,colorlinks=true}
\usepackage{commath}
\usepackage{amsmath}
\usepackage{float}
\usepackage{listings,xcolor}
\usepackage{cleveref}
\usepackage{afterpage}


	
\DeclareMathOperator*{\argmin}{arg\,min}
%\usepackage[pdftex,bookmarks,raiselinks,pageanchor,hyperindex,colorlinks]{hyperref}
\definecolor{highlight}{RGB}{230, 230, 230}
\newcommand*\NewPage{\newpage\null\thispagestyle{empty}\newpage}
\newcommand{\kb}{\textit{KB/s}}
\newcommand{\kbs}{\textit{Kbps}}
\newcommand{\mbs}{\textit{Mbps}}
\newcommand{\gbs}{\textit{Gbps}}
\newcommand{\mb}{\textit{MB/s}}
\newcommand\blankpage{%
\null
\thispagestyle{empty}%
\addtocounter{page}{-1}%
\newpage}
\geometry{
left=3cm, right=2cm,
top=2cm, bottom=2cm,
includefoot=true,
includehead=true
}
% create the header for this file
\pagestyle{fancy}

\usepackage{tikz}
\usetikzlibrary{fit}
\usetikzlibrary{arrows,automata}
\usetikzlibrary{arrows,shapes,positioning,shadows,trees}
\usepackage{booktabs}
\setlength{\heavyrulewidth}{1.5pt}
\setlength{\abovetopsep}{4pt}

\usetikzlibrary{positioning,arrows.meta,spy}

\definecolor{arrowblue}{RGB}{98,145,224}

\newcommand\ImageNode[3][]{
\node[draw=arrowblue!80!black,line width=1pt,#1] (#2) {\includegraphics[width=3cm,height=3cm]{#3}};
}
\renewcommand {\baselinestretch}{1.5}

\setcounter{secnumdepth}{4}
\setcounter{page}{2}
\begin{document}

\thispagestyle{empty}
\begin{center}
\begin{large}
\textbf{ĐẠI HỌC QUỐC GIA THÀNH PHỐ HỒ CHÍ MINH}
\end{large} \\
\begin{large}
\textbf{ĐẠI HỌC BÁCH KHOA}
\end{large} \\
\begin{large}
\textbf{KHOA KHOA HỌC VÀ KĨ THUẬT MÁY TÍNH}
\end{large} \\
\textbf{--------------------  *  ---------------------}\\[1.5cm]
\begin{center}
\includegraphics[scale=.2]{hinh/logo.png}
\end{center}

{\fontsize{26pt}{1}\selectfont \textbf{BÁO CÁO}}\\
{\fontsize{26pt}{1}\selectfont \textbf{LUẬN VĂN TỐT NGHIỆP}}\\[0.5cm]

{\fontsize{24pt}{1}\selectfont \textbf{Đề tài:} XÂY DỰNG ỨNG DỤNG\\
							ĐỌC TIN VÀ TRẢ LỜI ĐIỆN THOẠI BẰNG GIỌNG NÓI}\\[1.75cm]
\end{center}

\hspace{2cm} Hội đồng \hspace{74pt} :  \hspace{4pt} \textbf{\parbox[t]{10cm}{
Khoa Học và Kĩ Thuật Máy Tính
}} \\

\hspace{2cm} Giáo viên hướng dẫn \hspace{14pt} :  \hspace{4pt} \textbf{\parbox[t]{8cm}{
Phạm Hoàng Anh
}} \\



\hspace{2cm} Nhóm sinh viên thực hiện : \hspace{4pt}
\textbf{\parbox[t]{20cm}{
Võ Hoàng Ân \hspace{25pt} - 51300197
\\
Đặng Quốc Sơn \hspace{13pt} - 51303399
\\
}}


\vspace{1cm}
%	\vspace{1.75cm}
\begin{center}
{\fontsize{20pt}{1} TP Hồ Chí Minh}\\
{\fontsize{20pt}{1} \today}
\end{center}

% %%%%%%%%%%%%_ MUC LUC_%%%%%%%%%%%%%%%

\tableofcontents
\newpage
 
\section{Phân tích yêu cầu.}
	\subsection{Tính năng tin nhắn}
	\begin{itemize}
		\item Ứng dụng cần có quyền đọc,ghi tin nhắn.
		\item Ứng dụng có thể chuyển tin nhắn văn bản thành giọng nói.
		\item Ứng dụng có thể chuyển giọng nói thành văn bản.
		\item Ứng dụng cần có quyền gửi tin nhắn.
		\item Ứng dụng có thể đọc tên người gửi tin nhắn.
		\item Ứng dụng có thể  đọc, ghi và gửi tin nhắn trên các ứng dụng khác như messenger, zalo, viber.
		\item Ứng dụng có thể đọc mail và trả lời mail bằng giọng nói.
	\end{itemize}	
	\subsection{Tính năng gọi điện}
	\begin{itemize}
		\item Ứng dụng cần có quyền truy nhập danh ba.
		\item Ứng dụng có thể nhận diện giọng nói để nhận, từ chối cuộc gọi.
		\item Ứng dụng có thể đọc tên người gọi.
	\end{itemize}
\subsection{Tính năng khác}
\begin{itemize}
	\item Có thể hỗ trợ nhiều ngôn ngữ.
	\item Cho phép cá nhân hoá ứng dụng.
	\begin{itemize}
		\item Chỉ hoạt động khi có headphone được kết nối (kết nối có dây hoặc bluetooth).
		\item Định nghĩa các từ viết tắt, Ví dụ <3 sẽ được dịch thành "biểu tượng trái tim" khi chuyển sang giọng nói.
		\item Cho phép đọc tin nhắn khi ứng dụng nghe nhạc đang hoạt động.
		\item Có icon trên thanh thông báo, biểu thị ứng dụng đang chạy.
	\end{itemize}
\end{itemize}

\section{Công nghệ chuyển văn bản thành giọng nói}
	\begin{itemize}
		\item Sử dụng cloud api.\\
		Yêu cầu có kết nối internet.
			\begin{itemize}
				\item iSpeech API 
					\begin{itemize}
						\item Có phiên bản miễn phí và trả phí.
						\item Phiên bản miễn phí có popup quảng cáo khi sử dụng.
						\item Sử dụng SDK cung cấp để dùng API.
					\end{itemize}
				\item 	Translation API 
					\begin{itemize}
						\item Sản phẩm của Google Cloud Platform.
						\item Yêu cầu đăng kí khi sử dụng, miễn phí một năm đầu.
						\item Sừ dụng giao thức HTTP để dùng API.
					\end{itemize}
				\item Tạo server google-translate\\
				Sử dụng open source từ nguồn \href{url}{https://github.com/guyrotem/google- translate-server} , deloy lên heroku hoặc opensh.
			\end{itemize}
		\item Sử dụng SDK native\\
				Không yêu cầu kết nối internet.
				\begin{itemize}
					\item Android \\
						Sử dụng Text to speech API.
					\item IOS \\
						Sử dụng  AVSpeechSynthesizer API.
			\end{itemize}
	\end{itemize}

\section{Công nghệ chuyển giọng nói thành văn bản }
	\begin{itemize}
			\item Sử dụng cloud api.\\
			Yêu cầu có kết nối internet.
			\begin{itemize}
				\item CLOUD SPEECH API 
					\begin{itemize}
						\item Sản phẩm của Google Cloud Platform.
						\item Yêu cầu đăng kí khi sử dụng, miễn phí một năm đầu.
						\item Sử dụng REST or gRPC request.
					\end{itemize}
			\end{itemize}
			\item Sử dụng SDK native\\
			Không yêu cầu kết nối internet.
				\begin{itemize}
					\item Andoird \\
					Sử dụng RecognizerIntent API.
					\item IOS \\	
					Sử dụng Speech API.			
				\end{itemize}
	\end{itemize}
\section{Một số ứng dụng tương tự}
	\begin{itemize}
		\item ReadItToMe \\
		Tính năng
			\begin{itemize}
				\item Đọc và trả lời tin nhắn bằng giọng nói
				\item Đọc thông báo tới từ các ứng dụng khác như hangout hoặc WhatApp.
				\item Dung giọng nói trả lời SMS, WhatApp, FaceBook...
				\item Chỉ đọc khi có thiết bị bluetooth hay tai nghe được kết nối.
				\item Cho phép định nghĩa các từ viết tắt.
				\item Có thể đọc tin nhắn khi trình phát nhạc đang chạy.
				\item Đọc tên người gửi.
				\item Chỉ hỗ trợ Android
			\end{itemize}
		\item DriveSafe.ly \\
		Tính năng
		\begin{itemize}
			\item Đọc tin nhắn với nhiều lựa chọn, cho phép cá nhân hoá.
			\item Đọc tin nhắn và email theo thời gian thực và tự động trả lời mà không cần người sử dụng thao tác.
			\item Hỗ trợ Android và BlackBerry.
		\end{itemize}
		\item Text'nDrive \\
		Tính năng 
			\begin{itemize}
				\item Đọc email và trả lời theo giọng nói.
				\item Hỗ trợ IOS, BlackBerry OS
			\end{itemize}
	\end{itemize}

\end{document}
